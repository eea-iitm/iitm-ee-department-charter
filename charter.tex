\documentclass[12pt]{charter}
\makeatletter
\newcommand{\authortable}[1]{
  \gdef\@authortable{%
  \bgroup
	\def\arraystretch{1.5} %increase array border spacing
  \begin{tabular}{|c|c|}
  \hline
  \textsc{Rakesh Raavi} & Department Legislator (Academic) 2018-19\\
  \hline
  \textsc{Sudharshan R} & Research Affairs Secretary 2018-19\\
  \hline
  \textsc{Samyak Raj Pasala} & EEA Secretary 2018-19\\
  \hline
  \textsc{B Akhil} & EEA Core 2017-18\\
  \hline
  \end{tabular}
  \egroup
  }
}
% add the extra information after the date
\postdate{\par\vspace{-6pt}\@authortable\end{center}}
\makeatother

%To suppress toc warnings: bookmark level for unknown article defaults to 0.
\begin{document}
\title{Department Charter\\Department of Electrical Engineering\\IIT Madras}
\author{Drafting Committee:}
\authortable{} % line 2 for explanation and modification
\date{}
\maketitle
\setcounter{tocdepth}{1}
\tableofcontents
\newpage

\article*{Preamble}
We, the students of the Department of Electrical Engineering, IIT Madras, give to ourselves this charter to promote the welfare of the student community of our department and regulate the student bodies functioning within our department. Towards these ends, we resolve to respect and abide by this charter and the Students' Constitution of IIT Madras.


\article{General Provisions}

The authority of this Charter and the Students’ Constitution of IIT Madras are the basis for all activities conducted by the students of the Department of Electrical Engineering, IIT Madras.

\section{Membership}
All students and research scholars enrolled at the Department of Electrical Engineering are members of the Department Student Body (DSB) and shall be subject to this Charter and the laws made under the authority of this Charter.


\section{Political Power}
All political power granted to bodies authorized under this Charter is inherent in the Department Student Body. The rights of the Department Student Body, not otherwise limited by law established in and under the purview of this Charter or the Students’ Constitution of IIT Madras, supersede those of all other student organizations.

\section{Precedence}
This Charter is subservient to the Students’ Constitution of IIT Madras but within limits established in the Students’ Constitution, this Charter shall take precedence over all other instruments for the conduct of student activities of the Department of Electrical Engineering.

\section{Department Officials}
All elected Department Officials shall be members of the Department Student Body in good academic and ethical standing, as determined by the Student Election Commission (SEC), at the time of the selection and appointment. These officials are expected to maintain the highest levels of ethical standard during the term of office.

\section{Rights of the Department Student Body}
Every student of the Department of Electrical Engineering shall enjoy, at a minimum, the rights guaranteed by the Students’ Constitution of IIT Madras. This does not prevent the Department Student Body from guaranteeing under this Charter additional rights to students insofar as they do not violate the Students’ Constitution.

\section{Resolution of Conflicts and Interpretation of the Charter}
As far as possible, conflicts arising from the interpretation and implementation of this Charter shall be resolved amicably amongst students. However, in cases where such a resolution of conflict does not occur, the verdict of the Student Ethics and Constitutional Commission shall be final and binding.

\pagebreak

\article{Legislative}

\section{General Body Meetings}
The Department Student Body shall meet at a General Body Meeting (GBM) in order to discuss the issues of its interest, to pass policies, to review the performance of the members of the Department Council, and to make amendments to the Department Charter, if necessary.

\renewcommand{\labelenumi}{\Alph{enumi}}
\begin{enumerate}
	\item The Branch Councillor/Department Legislator (Academic) (or) Research Councillor (or) Department Legislator (Research) shall call for the General Body Meeting;
	\item There shall be at least one General Body Meeting every semester;
	\item The quorum for any GBM shall be 10 percent of the total current strength of the students of the Department;
	\item It is mandatory for all the members of the Department Council to attend the GBMs;
	\item The minutes of all GBMs shall be communicated to the Department Student Body within 72 hours of holding the GBM.
\end{enumerate}

Online polls conducted by Councillor(s)/Legislator(s) that are open to the Department Student Body and having the requisite quorum shall also be considered as a legitimate General Body Meeting.

\section{Amendments to the Charter}
Amendments to this charter can be proposed by any member of the Department Student Body by
following the procedure detailed below -

\begin{enumerate}
	\item Amendments to this Charter can only be considered in a physical General Body Meeting having at least 10\% quorum which is composed of at least 10 people from each academic program (B.Tech, Dual Degree, M.Tech, M.S, and Ph.D), and at least 5 people from each of the EE streams (EE1, EE2, and so on). Only MTech, MS, and PhD	students can be considered for satisfying the stream requirement.
	\item A mail must be circulated to all members of the Department Student Body by the proposer of the amendment detailing the need for the same at-least one week before the	physical General Body Meeting.
	\item The amendment shall be considered passed if the quorum for the physical General Body Meeting	is met and 2/3rd of students present and voting approve of the same. The amendment	motion shall be deferred to the next physical General Body Meeting in case of the aforementioned quorum not being met. The quorum for the next physical General Body Meeting where the amendment is considered shall be 15 percent of the total current strength of the students. The amendment shall be considered passed if this quorum is met and 2/3rd of students present and voting approve of the same.
\end{enumerate}

\pagebreak

\article{Executive}


\section{Department Council}

The Department Council shall be responsible for taking decisions consistent with the policies set in the General Body Meetings. It shall be in continuous touch with the Department Student Body and shall apprise it of issues concerning students’ interest. The members of the Department Council are:

\begin{enumerate}
	\item Branch Councillor/Department Legislator (Academic)
	\item Research Councillor
	\item Department Legislator (Research)
	\item M.Tech Representative
	\item Electrical Engineering Association (EEA) Secretary
	\item Class Representatives
	\item Placement Cores (invitees)
	
\end{enumerate}

Only students of the Department of Electrical Engineering, irrespective of the course or year that they are in, can become a member of the Department Council.

\subsection{The Councillors}

The Councillors (Branch Councillor/Research Councillor) shall be elected representatives and shall be the Heads of the Department Council. All other Department Officials shall report and be accountable to the Councillors. There shall be two meetings in a semester where other Department Officials shall report to the Councillors. These meetings shall be open to the Department Student Body. The eligibility criteria, duties and responsibilities of the Councillors are as enumerated in the Students’ Constitution.

\subsection{M.Tech Representative}

The M.Tech Representative shall be an M.Tech student. The M.Tech Representative shall be elected by the M.Tech students of the Department of Electrical Engineering, in an election conducted by a committee consisting of Branch Councillor/Department Legislator (Academic), Research Councillor and Department Legislator (Research). The eligibility criteria and the election procedure for this position shall be specified by the above committee.

\subsubsection{Duties and Responsibilities}
The M.Tech Representative shall-
\begin{enumerate}
	\item Represent any academic/research concerns of the students to the Head of the Department and the Department Consultative Committee and shall facilitate interactions between faculty and students;
	\item Attend the Department Consultative Committee meetings and represent the views of the students;
	\item Help students tackle their specific academic/research problems and take up their problems with the respective authorities;
	\item Be conversant with all the rules and procedures of the Department.
	
\end{enumerate}

\subsection{Electrical Engineering Association (EEA) Secretary}
\subsubsection{Selection}
The Department Secretary shall be selected by the outgoing Department Secretary in consultation with the Branch Councillor/Department Legislator (Academic) and the Research Councillor through a transparent and well-publicized application process.

\subsubsection{Duties and Responsibilities}
The EEA Secretary shall -
\begin{enumerate}
	\item Conduct at least one sports tournament per semester and at least two unique sports tournaments per year. Online games and board games cannot be considered for meeting this requirement;
	\item Be responsible for conducting technical events like tutorials, lectures and workshops on Electrical Engineering or related topics. At least three technical events per semester should be conducted;
	\item Conduct at least two talks per semester delivered by experienced electrical engineers;
	\item Make an effort to conduct department trips and field visits;
	\item Conduct any other extra and co-curricular activities in the department, including, but not limited to Freshie Night and Department Quizzes;
	\item Be responsible for promoting intra-department bonding.
	
\end{enumerate}

The ultimate responsibility of ensuring that the Department Secretary’s duties and responsibilities are fulfilled lies with the Department Secretary alone.

\subsection{Class Representatives}
\subsubsection{Selection}
The Class Representatives shall be elected by their class in the elections conducted by the Branch
Councillor/Department Legislator (Academic).

\subsubsection{Duties and Responsibilities}
The Duties and Responsibilities of the Class Representatives are enumerated in the Students’
Constitution. The class representatives shall report to the Branch Councillor/Department Legislator (Academic) for the satisfactory execution of their duties and responsibilities.

\pagebreak

\article{Electrical Engineering Association (EEA)}

\section{Purpose}

The primary purpose of the Electrical Engineering Association (EEA) is to develop technical skills which are not part of the curriculum among the student community of the Electrical Engineering Department. EEA shall also strive to promote intra-department bonding among the student community. EEA shall be advised by a Faculty Adviser appointed by the Head of Department of the Electrical Engineering Department. To be eligible for any position in EEA, a person should be a student of the Electrical Engineering Department and shall have a minimum of 6.5 CGPA with no Us and no Ws at the time of selection.

\section{Structure}


\begin{enumerate}
	\item Secretary
	\item Core team
	\item Coordinators
\end{enumerate}

\subsection{Core Team}
The core team shall be selected by the secretary in consultation with the Faculty Adviser. The strength of the core team shall not exceed three. The division of work among the core team shall the communicated by the secretary to the core team at the start of the academic year.

\subsection{Coordinators}
The cores shall recruit coordinators to help them out with their duties. The total number of coordinators in EEA shall not exceed eight. The number of coordinators each core can recruit shall be specified by the secretary.

\section{Dismissal of EEA Secretary}
The EEA secretary can be dismissed by passing a simple majority vote in a physical General Body Meeting. The dismissal motion shall be communicated to the Department Student Body at least 72 hours prior to the General Body Meeting. If the quorum cannot be met, the secretary can be dismissed by the councillors with approval from the Faculty Adviser and Head of the Department of the Electrical Engineering Department.\\
In the event of a secretary's dismissal, the core team shall internally elect a secretary in consultation with the Faculty Adviser for the rest of the academic year.

\section{Dismissal of Cores and Coordinators}
The cores can be dismissed by the secretary in consultation with the Faculty Adviser. The grounds for removal can only be unsatisfactory performance or misconduct within the team. The decision shall be communicated to the concerned core by the secretary in writing.\\
The coordinators can be dismissed by the core team in consultation with the secretary. The grounds for removal can only be unsatisfactory performance or misconduct within the team. The decision shall be communicated to the concerned coordinator by the secretary in writing.

\section{Finance}
The EEA bank account shall be jointly held by the Faculty Adviser and the secretary with the cheque book maintained by the Faculty Adviser and the online banking credentials (with view rights only) held by the secretary. An annual financial statement detailing all transactions conducted in the academic year shall be communicated to the Department Student Body before the start of the next academic year.\\
EEA can collect money from members of the IIT Madras Student Body only for specific events or purposes. Students can choose to opt out of paid events.

\pagebreak

\article*{\normalfont\scshape\Large Annexure A: List of amendments to the Department Charter}
\addcontentsline{toc}{article}{Annexure A: List of amendments to the Department Charter}

\begin{table}[h]
	\begin{center}
		\bgroup
		\def\arraystretch{1.5}
		\begin{tabularx}{\textwidth}{|c|c|Y|Y|}
			\hline
			S. No. & Date Amended & Original Clause & Amendment\\
			\hline
			% add new rows by copy-pasting the previous two lines (S.No. and \hline) and replacing the text
		\end{tabularx}
		\egroup
	\end{center}
\end{table}
\end{document}